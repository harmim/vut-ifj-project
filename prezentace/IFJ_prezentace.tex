% Author: Vojtěch Hertl

\documentclass[11pt, hyperref={unicode}]{beamer}

\usepackage[czech]{babel}
\usepackage[utf8]{inputenc}
\usepackage{times} % font
\usepackage{graphics} % vkládání obrázků

% téma prezentace
% metropolis progressbar bug - kvuli tomu je na každém snímku varování Overfull \hbox
\usetheme[progressbar=frametitle]{metropolis}
% číslování aktuální snímek/celkem snímků
\setbeamertemplate{footline}[frame number]


\title{Formální jazyky a překladače}
\subtitle{
	Implementace překladace imperativního jazyka IFJ17.
	Tým 104, varianta II.
}
\author[Dominik Harmim \& Timotej Halás \& Matej Karas \& Vojtěch Hertl]
{
	\texorpdfstring{
		\begin{columns}
			\column{.45\linewidth}
			\centering
			Dominik Harmim\\
			\footnotesize{\texttt{xharmi00@stud.fit.vutbr.cz}}
			\column{.45\linewidth}
			\centering
			Timotej Halás\\
			\footnotesize{\texttt{xhalas10@stud.fit.vutbr.cz}}
		\end{columns}
		\vspace{0.5cm}
		\begin{columns}
			\column{.45\linewidth}
			\centering
			Matej Karas\\
			\footnotesize{\texttt{xkaras34@stud.fit.vutbr.cz}}
			\column{.45\linewidth}
			\centering
			Vojtěch Hertl\\
			\footnotesize{\texttt{xhertl04@stud.fit.vutbr.cz}}
		\end{columns}
		\vspace{1cm}
	}
	{Dominik Harmim \& Timotej Halás \& Matej Karas \& Vojtěch Hertl}
}
\date{\today}
\institute
{
	Vysoké učení technické v~Brně\\
	Fakulta informačních technologií

	\bigskip

	\scalebox{0.3}{\includegraphics{img/FIT_barevne_CMYK_CZ.eps}}
}


\begin{document}



\maketitle



\begin{frame}{Přehled}
	% číslovaný obsah
	\setbeamertemplate{section in toc}[sections numbered]
	% skrytí podsekcí v obsahu
	\tableofcontents[hideallsubsections]
\end{frame}



\section{Struktura programu}

\begin{frame}
	\frametitle<1>{Zdrojový kód}
	\frametitle<2>{Hlavní program}
	\frametitle<3>{Syntaktický analyzátor}
	\frametitle<4>{Lexikální analyzátor}
	\frametitle<5>{Výrazy}
	\frametitle<6>{Generování kódu}


\begin{overprint}
 	\onslide<1>\centerline{\includegraphics[width=0.95\linewidth]{img/zdrojovy_kod.pdf}}%
 	\onslide<2>\centerline{\includegraphics[width=0.95\linewidth]{img/main.pdf}}%
 	\onslide<3>\centerline{\includegraphics[width=0.95\linewidth]{img/parser.pdf}}%
 	\onslide<4>\centerline{\includegraphics[width=0.95\linewidth]{img/scanner.pdf}}%
 	\onslide<5>\centerline{\includegraphics[width=0.95\linewidth]{img/vyrazy.pdf}}%
 	\onslide<6>\centerline{\includegraphics[width=0.95\linewidth]{img/generovani_kodu.pdf}}%
\end{overprint}
\vspace{-10cm}
\begin{overprint}
  	\onslide<1> 
  	\begin{itemize}
		\item IFJ17.
	\end{itemize}
 	\onslide<2>
  	\begin{itemize}
		\item Nastavení vstupního souboru.
		\item Zahájení syntaktické analýzy.
		\item Generování kódu nebo vracení chyby na výstup.
	\end{itemize}
  	\onslide<3>
  	\begin{itemize}
		\item Volání lexikální analýzy pro každý token.
		\item Implementace podle LL gramatiky a LL tabulky.
		\item Volání generování vnitřního kódu.
		\item Tabulka symbolů.
	\end{itemize}
    \onslide<4>
    \begin{itemize}
		\item Implementace podle konečného automatu.
		\item Načítá ze vstupu a vrací token.
	\end{itemize}
    \onslide<5>
    \begin{itemize}
		\item Precedenční syntaktická analýza.
		\item Implementace podle precedenční tabulky.
		\item Sémantická analýza.
		\item Volání generování matematických a logických instrukcí.
	\end{itemize}
    \onslide<6>
    \begin{itemize}
		\item Generování mezikódu IFJcode17.
		\item Generování vestavěných funkcí.
	\end{itemize}
\end{overprint}

\end{frame}

\section{Práce v~týmu}

\begin{frame}{Práce v~týmu}
	\begin{alertblock}{GitHub}
		\begin{itemize}
			\item GitHub jako vzdálený repositář.
			\item Práce ve své větvi.
		\end{itemize}
	\end{alertblock}
	\begin{alertblock}{Slack}
		\begin{itemize}
			\item Tématicky rozdělné skupinové konverzace.
			\item Automatické notifikace z~GitHubu.
			\item Upozornění na blížící se deadline.
		\end{itemize}
	\end{alertblock}
	
\end{frame}

\section{Dotazy}

\begin{frame}{Dotazy}
	\vspace{-5cm}
	\Large{Děkuji za pozornost, prostor pro Vaše dotazy.}
\end{frame}
\end{document}
